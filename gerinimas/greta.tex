\begin{sritiesGerinimas}{Requirements Management}
    \vertinimas{0}{1}
    \veiksmas{
        \begin{tiksloGerinimas}{REQM.SG 1 Manage Requirements}
        \vertinimas{Unsatisfied}{Satisfied}
        \veiksmas{
            \begin{praktikosGerinimas}{REQM.SP 1.1. Understand Requirements}
                \vertinimas{PI}{LI}
                \veiksmas{Sukuriamas naujas produktas: 
                \hspace*{1cm}\newline Id:
                \hladd{KR}
                \hspace*{1cm}\newline Pavadinimas:
                \hladd{Kliento reikalavimai projektui}
                \hspace*{1cm}\newline Aprašymas:
                \hladd{
                Kliento pusėje suprojektuoti reikalavimai projektui, kurie apibrėžia detales, kaip kliento poreikiai projektui (KP) turėtų būti išpildyti projekte. 

                Kliento poreikių projektui (KP) dokumentas apima techninius, kokybinius, ir veikimo reikalavimus, reikalingus projekto sėkmei užtikrinti. 
                %Reikalavimų kokybės vertinimo kriterijaii!!! Sudaroma atsekamumo matrica. %\

                %Reikalavimų valdymas: tiesiog naują proc reiks padaryti%
                }
                
                }

                \veiksmas{Pridėti produktą prie Kliento įtraukimas (KĮ) proceso "Panaudoti darbo produktai" skilties }
                \veiksmas{Kliento įtraukimo (KĮ) procese pakeisti 1. veiklą ją išdėstant taip:  


                Projektų vadovas, architektas ir analitikas bendrauja su klientu, aiškinasi jo poreikius projektui
                (KP)  
                \hladd {ir kliento reikalavimus projektui (KR)}
                . Ši veikla tęsiasi tol, kol įmonės atstovai surenka pakankamai informacijos paruošti klientui
                pasiūlymą}
                

                \veiksmas{Kliento įtraukimo (KĮ) procese pakeisti 2. veiklą ją išdėstant taip:
                
                Projektų vadovas, architektas ir analitikas paruošia pradinę projekto viziją tam, kad būtų patvirtintas projekto įgyvendinamumas ir suderinta bendra projekto kryptis su SŠ. Jie taip pat atlieka
                pagrįstumo analizę, siekdami įvertinti, ar kliento poreikiai (KP) \hladd{ir kliento reikalavimai produktui (KR)} yra įgyvendinami.}

                \veiksmas{Kliento įtraukimo (KĮ) procese pakeisti 5. veiklą ją išdėstant taip:

                Klientui yra pateikiama sutartis (KS). Jei klientas yra patenkintas sutarties sąlygomis, pereinama
                prie sutarties pasirašymo 6. Klientas gali nesutikti su sutarties sąlygomis. Tokiu atveju vyksta
                derybos - klientas pateikia naujus poreikius (KP) ir \hladd{ kliento reikalavimus projektui (KP), ir} dar kartą vykdoma veikla 2. Jei abi šalys
                nesugeba rasti kompromiso, procesas gali būti nutrauktas ir darbas su klientu netęsiamas.
                }
                
                \argumentacija{ Atlikus šiuos pakeitimus, proceso specifinės praktikos įvertinimas gali kilti iki \textbf{LI} įvertinimo. Išlieka trūkumas, tik toks, jog galbūt ne visoms kompanijoms tiks departamento strategija, jog kliento pusėje renkami reikalavimai nebus vertinami, bet mes įvertinome, jog tai nėra esminis trūkūmas.}
            \end{praktikosGerinimas}
        }


        
        \veiksmas{
            \begin{praktikosGerinimas}{REQM.SP 1.2. Obtain Commitment to Requirements}
                \vertinimas{PI}{LI}
                \veiksmas{Sukurti naują procesą Reikalavimų valdymas (RV), jį įterpti po 2.5.4. procesu ir įdėti į procesų aprašyumo BPMN diagramą:
                \hspace*{1cm}\newline Pavadinimas: \hladd{Reikalavimų valdymas}   
                \hspace*{1cm}\newline Sutrumpinimas: \hladd{RV}
                \hspace*{1cm}\newline Tikslas: \hladd{Sekti funkcinių ir nefunkcinių reikalavimų pokyčius ir kokybę.}
                \hspace*{1cm}\newline Panaudoti darbo produktai: 
                \hladd{
                    
                    • FR. Funkciniai reikalavimai
                    
                    • NFR. Nefunkciniai reikalavimai
                    
                    • GRR. Grįžtamojo ryšio registras

                    • RVP. Rizikų valdymo planas
                    
                }
                \hspace*{1cm}\newline Sukurti darbo produktai: \hladd{
                
                    • FR. Funkciniai reikalavimai
                    
                    • NFR. Nefunkciniai reikalavimai
                    
                    • GRR. Grįžtamojo ryšio registras
                
                }
                \hspace*{1cm}\newline Veiklos: \hladd{

                1.Architektas ir analitikas pagal grįžtamojo darbo registrą (GRR) apmąsto, kaip gali keistis funkciniai reikalavimai (FR) ir nefunkciniai reikalavimai (NFR). 
                
                2.Pagal rizikų valdymo planą (RVP) įvertina reikalavimų pokyčių rizikas. 
                
                3. Klientai informuojami apie rizikas. Komandos nariai išklauso SŠ atsiliepimus apie rizikas ir  projekto vadovas surašo surinktus atisiliepimus į grįžtamojo ryšio registrą (GRR). 

                4. Architektas ir analitikas Pakeičia funkcinius reikalavimai (FR) ir nefunkcinius reikalavimus (NFR). Jų pokyčiai ir pokyčių priežastys užregistruojami ir laikomi "Jira" platformoje.
                
                5.Architektas ir analitikas sudaro atsekamumo matricą funkcinių reikalavimų (FR) ir nefunkcinių reikalavimų (NFR) pokyčių kokybei užtikrinti. Jei trūkumų yra, vėl vykdoma 3 ir 4 veikla.
                
                6.Projektų vadovas įvertina ar funkcinių reikalavimų (FR) ir nefunkcinių reikalavimų (NFR) pokyčiai gali pakeisti procesuose esančius darbo produktus ir veiklas, jei gali, jie yra pakeičiami.
                }
                }
                \argumentacija{ Atlikus šiuos pakeitimus, proceso specifinės praktikos įvertinimas gali kilti iki \textbf{LI} įvertinimo. Išlieka trūkumas, tik toks, jog nėra vykdomos derybos dėl departamente paskirstytų išteklių.}
            \end{praktikosGerinimas}
        }


        
        \veiksmas{
            \begin{praktikosGerinimas}{REQM.SP 1.3 Manage Requirements Changes}
                \vertinimas{PI}{FI}
                 \argumentacija{Atlikus -REQM.SP 1.2. Obtain Commitment to Requirements specifiniai praktikai pagerinti reikalingus veiksmus, proceso specifinės praktikos įvertinimas gali kilti iki \textbf{FI} įvertinimo. Išlieka trūkumas, tik toks, jog nėra vykdomos derybos dėl departamente paskirstytų išteklių.}
            \end{praktikosGerinimas}
        }


        \veiksmas{
            \begin{praktikosGerinimas}{REQM.SP 1.4 Maintain Bidirectional Traceability of Requirements}
                \vertinimas{NI}{LI}
                \argumentacija{Atlikus -REQM.SP 1.2. Obtain Commitment to Requirements specifiniai praktikai pagerinti reikalingus veiksmus, proceso specifinės praktikos įvertinimas gali kilti iki \textbf{LI} įvertinimo. Išlieka trūkumas, tik toks, jog sekami žemesnio ir aukštesnio lygio reikalavimai, mes įvertinom, jog tai nėra esminis pokytis pasiekti LI.}
            \end{praktikosGerinimas}
        }



        \veiksmas{
            \begin{praktikosGerinimas}{REQM.SP 1.5 Ensure Alignment Between Project Work and Requirements}
                \vertinimas{NI}{FI}
                \argumentacija{
                    Atlikus -REQM.SP 1.2. Obtain Commitment to Requirements specifiniai praktikai pagerinti reikalingus veiksmus, proceso specifinės praktikos įvertinimas gali kilti iki \textbf{FI} įvertinimo. 
                }
            \end{praktikosGerinimas}
        }

        
        \argumentacija{
            Atlikus šiuos pakeitimus, visos specifinio tikslo praktikos turės įvertinimą ≥ LI, o likę nustatyti trūkumai
            (nėra vykdomos derybos dėl papildomų išteklių, reikalavimai nėra skirstomi į žemesnio lygio reikalavimus) neturės esminio poveikio. Todėl specifinis tikslas laikomas pasiektu (\textbf{Satisfied}).
        }
        \end{tiksloGerinimas}
    }
    \argumentacija{
        Atlikus šiuos pakeitimus, visi Requirement Management specifiniai    tikslai bus įvertinti kaip \textbf{Satisfied}, todėl
        proceso srities gebėjimo lygis pakils iki 1.
    }
\end{sritiesGerinimas}











% \begin{sritiesGerinimas}{Process and Product Quality Assurance}
%     \vertinimas{0}{1}
%     \veiksmas{
%         \begin{tiksloGerinimas}{PPQA.SG 1 Objectively Evaluate Processes and Work Products}
%         \vertinimas{Unsatisfied}{Satisfied}
%         \veiksmas{
%             \begin{praktikosGerinimas}{PPQA.SP 1.1 Objectively Evaluate Processes}
%                 \vertinimas{NI}{LI}
%                 \veiksmas{Citata, kuri eina i pagerinta procesu doc'a}
%                 \argumentacija{Praktika pagerinta args.}
%             \end{praktikosGerinimas}
%         }
%         \veiksmas{
%             \begin{praktikosGerinimas}{PPQA.SP 1.2 Objectively Evaluate Work Products}
%                 \vertinimas{NI}{LI}
%                 \veiksmas{Citata, kuri eina i pagerinta procesu doc'a}
%                 \argumentacija{Praktika pagerinta args.}
%             \end{praktikosGerinimas}
%         }
%         \argumentacija{Tikslo pagerinimo argumentacija}
%         \end{tiksloGerinimas}
%     }
%     \argumentacija{Srities pagerinimo argumentacija}
% \end{sritiesGerinimas}