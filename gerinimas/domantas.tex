\newpage

Project monitoring and control
\begin{sritiesGerinimas}{Project monitoring and control}
    \vertinimas{0}{1}
    \veiksmas{

        \begin{tiksloGerinimas}{SG 1 Monitor the Project Against the Plan}
        \vertinimas{Unsatisfied}{Satisfied}
        \veiksmas{

            \begin{praktikosGerinimas}{Monitor Stakeholder Involvement}
                \vertinimas{NI}{FI}
                \veiksmas{
                    Pridėti naują darbo produktą: 
                    \hladd{
                        \textbf{SŠĮA} -- Suinteresuotų šalių įsitraukimo ataskaita. Čia aprašomas SŠ įsitraukimas, susįjusios problemos ir jų pasekmės ir galimi sprendimo būdai.
                    }
                }
                \veiksmas{
                    Pristatymo ir grįžtamojo ryšio surinkimo (PAS) procese pridėti veiklą: 
                    \hladd{
                        Projekto vadovas peržiūti suinteruotų šalių įsitraukimą. Jeigu kyla problemų, atlieka analizę ir nustato jų poveikį. Tai, projektų vadovas, surašo suinteresuotų šalių įsitraukimo ataskaitoje.
                    }
                }
                \veiksmas{Kontrolės (KO) procese pakeisti 3 veiklą:
                    Projekto vadovas vykdo stebėseną ir esant reikalui švelnina rizikas arba vykdo žalos kontrolę pagal rizikų valdymo planą (RVP) \hladd{ir suinteresuotų šalių įsitraukimo ataskaitą (SŠĮA).}
                    }
                \argumentacija{Šie pakeitimai užtikrina, kad vyskta SŠ įsitraukimo kontrolė bei kad SŠ įsitraukimo kontrolės rezultatai yra dokumentuojami.}
            \end{praktikosGerinimas}

            \begin{praktikosGerinimas}{Conduct Progress Reviews}
                \vertinimas{PI}{LI}
                \veiksmas{Kontrolės (KO) procese pakeisti 3 veiklą:
                    \hladd{Analitikas analizuoja naują informaciją iš komandos narių atsiliepimų ir grįžtamojo ryšio registro, įvertina ar nepasikeitė esamos ir/ar neatsirado naujos projekto rizikos, pagal tai analizuoja rizikas ir atnaujina rizikų valdymo planą (RVP).}
                    Projekto vadovas vykdo stebėseną ir esant reikalui švelnina rizikas arba vykdo žalos kontrolę pagal rizikų valdymo planą (RVP) \hladd{, papildomai informuodamas atitinkamas SŠ.}
                    }
                \argumentacija{Šie pakeitimai užtikrina, kad rizikos yra nuolat analizuojamos, reaguojama į naują informaciją ir rizikas, taip pat informuojamos SŠ.}
            \end{praktikosGerinimas}

            \begin{praktikosGerinimas}{Monitor Project Planning Parameters}
                \vertinimas{PI}{LI}
                \veiksmas{Kontrolės (KO) procese pridėti veiklą:
                        Projekto vadovas peržiūri sprinto užduočių sąrašą (SUS), į jį atsižvelgdamas, palygina faktinį užduotims skirtus žmogiškuosius išteklius su pradinėmis sąmatomis (IS), pagal tai paskaičiuoja kainos ir įdėtų pastangų ataskaitą.
                    }
                \veiksmas{Kontrolės (KO) procese pakeisi veiklą 4:
                    Remiantis grįžtamojo ryšio registru (GRR)\hladd{, kainos ir įdėtų pastangų ataskaita} ir laiko valdymo ataskaita projekto vadovas atlieka projekto užduočių sąrašo (PUS) atnaujinimą, prireikus, užduočių prioritetų keitimą, pasakojimo vienetų intervalo (PVI) patikslinimą. Šie pakeitimai fiksuojami sprinto peržiūros ataskaitoje (SPA).
                }
                \argumentacija{Šie pakeitimai užtikrina, kad bus sekama kaina ir pastangos.}
            \end{praktikosGerinimas}

            \begin{praktikosGerinimas}{Conduct Progress Reviews}
                \vertinimas{PI}{LI}
                \veiksmas{Įgyvendinimo (ĮG) procese pakeisti veiklą 1:
                    Kiekvieną dieną komanda rengia trumpą Stand-Up susitikimą, kad aptartų progresą, nustatytu kliūtis ir sinchronizuotu ̨ veiklas tarp komandos narių. Šis susitikimas įprastai trunka ne ilgiau kaip 15 minučių ir susideda iš trijų pagrindinių punktų: kas buvo padaryta vakar, kas bus daroma šiandien, ir su kokiomis kliūtimis susiduriama.
                    \hladd{Projekto vadovas atsižvelgdamas į kliutis, jeigu reikia, atitinkamai pakeičia sprinto užduočių sąrašą (SUS).}
                }
                \argumentacija{TODO}
            \end{praktikosGerinimas}
            
        }
        \end{tiksloGerinimas}

%%%%%%%%%%%%%%%%%%%%%%%%%%%%%%%%%%%%%%%%%%%%%%%%%%%%%%%%%%%%%%%%%%%

        \begin{tiksloGerinimas}{PMC.SG 2 Manage Corrective Action to Closure}
        \vertinimas{Unsatisfied}{Satisfied}
        \veiksmas{
            \begin{praktikosGerinimas}{Take Corrective Action}
                \vertinimas{PI}{LI}
                \veiksmas{Kontrolės (KO) procese pakeisti 3 veiklą:
                    Projekto vadovas vykdo stebėseną ir esant reikalui švelnina rizikas arba vykdo žalos kontrolę pagal rizikų valdymo planą, \hladd{prieš tai peržiūrėdamas paveiktas SŠ ir gaudamas leidimą iš jų}.
                }
                \argumentacija{Šie pakeitimai užtikrina, kad gaunamas sutikimas švelninti rizikas arba vykdyti žalos kontrolę, gaunant sutikimą iš SŠ}
            \end{praktikosGerinimas}
        }
        \vertinimas{Unsatisfied}{Satisfied}
        \veiksmas{
            \begin{praktikosGerinimas}{Manage Corrective Actions}
                % 1. Monitor corrective actions for their completion. 
                % 2. Analyze results of corrective actions to determine the effectiveness of the corrective actions. 
                \vertinimas{PI}{LI}
                \veiksmas{Kontrolės (KO) procese pakeisti 3 veiklą:
                    Projekto vadovas vykdo stebėseną ir esant reikalui švelnina rizikas arba vykdo žalos kontrolę pagal rizikų valdymo planą (RVP) \hladd{, papildant rizikų valdymo planą koregavimo veiksmų eigos statusu}.
                }
                \veiksmas{Kontrolės (KO) procese pridėti veiklą:
                    Projekto vadovas peržiūri visų koregavimo veiksmų statusus. Peržiūri ar visi koregavimo veiksmai buvo tinkami ir pasiekė savo tikslą.
                }
                \argumentacija{Šie pakeitimai užtikrina, kad koregavimo veiksmai yra ne tik tinkamai stebimi, bet ir vertinamas jų veiksmingumas.}
            \end{praktikosGerinimas}
        }
        \end{tiksloGerinimas}
    }
\end{sritiesGerinimas}

\newpage
\begin{sritiesGerinimas}{Technical solution}
    \vertinimas{0}{1}
    \veiksmas{

        \begin{tiksloGerinimas}{SG 1 Select Product Component Solutions}
        \vertinimas{Unsatisfied}{Satisfied}
        \veiksmas{

            \begin{praktikosGerinimas}{Develop Alternative Solutions and Selection Criteria}
                \vertinimas{PI}{LI}
                \veiksmas{Reikalavimų analizės (RA) procese pridėti veiklą:
                    \hladd{Architektas ieško naujų technologijų ir COTS produktų, kuriuos galėtų panaudoti projekte, jeigu tokius atranda, papildo aukšto lygio architektūrą (ALSA).}
                }
                \argumentacija{Šis pakeitimas, jog atsižvelgiama į COTS produktus ir naujas technologijas, kurias būtų galima panaudoti projekte.}
            \end{praktikosGerinimas}
        }
        \end{tiksloGerinimas}
    }
\end{sritiesGerinimas}