\begin{sritiesGerinimas}{Validation}
    \vertinimas{0}{1}
    \veiksmas{
        \begin{tiksloGerinimas}{VAL.SG 1 Prepare for Validation}
        \vertinimas{Unsatisfied}{Satisfied}
        \veiksmas{
            \begin{praktikosGerinimas}{VAL.SP 1.2 Establish the Validation Environment}
                \vertinimas{PI}{FI}
                \veiksmas{Aplinkos rengimo (AR) procese pridėti veiklą: \hladd{Architektas su testuotoju atsižvelgia į kliento poreikius projektui ir identifikuoja testavimo įrankius, kurie gali priklausyti nuo kliento operacinės aplinkos, būtinus testavimui ir dokumentuoja repozitorijos apraše.}}
                \argumentacija{Atlikus šiuos pakeitimus bus išpildytos detaliosios praktikas, išskyrus detalų resursų suplanavimą. Todėl specifinė praktika įgyja FI įvertinimą.}
            \end{praktikosGerinimas}
        }
        \veiksmas{
            \begin{praktikosGerinimas}{VAL.SP 1.3 Establish Validation Procedures and Criteria }
                \vertinimas{PI}{FI}
                \veiksmas{Pridėti veiklą prie aplinkos rengimo (AR) ir produkto Aplinka (AP): \hladd{Architektas kartu su testuotojais apibrėžia operacinius scenarijus, aplinką, procedūras, įvestis ir išvestis, kurie tikrins realaus pasaulio sąlygas (našumą, saugumą ir pan.). Ši dokumentacija yra repozitorijos dalis.}}
                \veiksmas{Padengta nauju procesu \hladd{Reikalavimų valdymas (RV)}}
            \end{praktikosGerinimas}
        }
        \argumentacija{Atlikus pakeitimus, visų tikslo specifinių praktikų įvertinimai bus $\geq$ LI, o likę trūkumai (SŠ įtraukimas renkantis validacijos strategijas) nedarys esminio poveikio specifiniam tikslui. Todėl specifinis tikslas laikomas pasiektu (\textbf{Satisfied}).}
        \end{tiksloGerinimas}
    }
    \argumentacija{Atlikus šiuos pakeitimus, visi Validacijos specifiniai tikslai bus įvertinti kaip \textbf{Satisfied}, todėl proceso srities geb˙ejimo lygis pakils iki \textbf{1}}
\end{sritiesGerinimas}
