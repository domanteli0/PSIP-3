\section{Įmonės aprašymas}

\subsection{Įmonės pavadinimas}
„PTN“

% \subsection{"PTN" departments}

% \begin{enumerate}
%     \item  IT division
% \end{enumerate}

\subsection{Įmonės aprašymas}

% \textbf{NOTE:} Reikia aprašyti TIK IT depertamentą \newline

„PTN“ yra projektinė įmonė. „Produktų vystymas“ yra įmonės „PTN“ departamentas, kuris užsiima e.komercijos sistemų kūrimu klientams. „Produktų vystymo“ departamente įdarbinti apie 30 darbuotojų, t.\,y. šis skaičius gali šiek tiek svyruoti. Departamentas yra išskirstytas į tris komandas po maždaug 10 darbuotojų. Kiekviena komanda paraleliai dirba prie skirtingų projektų.

\subsection{Organizacinė struktūra}
Šiame dokumente modeliuojama „Produktų vystymo“ departamento komandų veikla.
\begin{table}[h!]
\centering
\begin{tabular}{p{0.1\textwidth}|p{0.9\textwidth}}
\hline
\textbf{Rolės} & \textbf{Atsakomybės} \\ \hline


% product developement & 

Projektų vadovas &

% \st{Manages and supervises the project, including setting project goals, timelines, and budgets. The project manager ensures that the development team meets deadlines and that the project is delivered successfully within the agreed-upon scope. They also act as the main point of communication between the client and the development team.}

Valdo ir prižiūri projektą, dalyvauja nustatant projekto tikslus, terminus ir biudžetą. Projekto vadovas užtikrina, kad programinės įrangos kūrimo  komanda laikytųsi terminų ir kad projektas būtų sėkmingai įgyvendintas sutarta apimtimi. Projektų vadovas komunikuoja su suinteresuotomis šalimis (SŠ).
\\ \hline
 % Deployment & Integrate developed systems with existing client platforms, for example: inventory or payment gateways. Maintain the internal IT infrastructure of the client companies. \\ \hline
Programinės įrangos kūrėjas &

% \st{ Responsible for developing the custom e-commerce software in line with the project's specifications. They ensure the functionality of the system through testing and work closely with the project manager to make sure client requirements are met. Developers also address all issues or bugs that arise during the development process. }

Atsakingas už programinės įrangos kūrimą pagal projekto tikslus.
\\ \hline

Testuotojas &

% \st{Responsible for evaluating the functionality and quality of software. They conduct various types of testing to identify defects before the product is delivered and work closely with developers to validate fixes to their reported defects. }

Atsakingas už programinės įrangos kokybės įvertinimą. Testuotojas atlieka įvairių tipų testavimą, kad nustatytų defektus prieš pristatant produktą klientui ir glaudžiai bendradarbiauja su programinės įrangos kūrėjais, kad patvirtintų praneštų klaidų pataisas.
\\ \hline

Architektas &

% \st{Designs the overall structure of the e-commerce software. The architect collaborates with developers to guide implementation and resolve complex technical challenges. }

Dalyvauja kuriant programinės įrangos architektūrą. Architektas bendradarbiauja su programinės
įrangos kūrėjais, kai kuriama programinė įranga ir padeda išspręsti sudėtingus techninius iššūkius.
\\ \hline
Analitikas & 

% \st{Gather and analyse client requirements to ensure the project meets business needs.}
Dalyvauja renkant ir analizuojant SŠ poreikius, juos dokumentuoja.
\\ \hline

% Paminėti, kad garantiniai atvejai

\end{tabular}
%\caption{Organizational Structure}
%\label{table:organizational_structure}
\end{table}


% \subsection{IT division}

% The IT department employs around 30 people. It carries out projects for private and public organisations and develops and maintains web applications in various fields.

% Short department description

% Project-based organisational structure

% E-commerce

% Agile/SCRUM

% Develops products and transfers them to the customer

% \subsubsection{Additional information}

% \paragraph{Organisational structure}
% \paragraph{Technologies used}
% \paragraph{Methodologies}
